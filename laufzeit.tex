Die Programmlaufzeit steigt mit der Entfernung zwischen Startjahr der Berechnung und dem Ergebnisjahr in etwa linear, da die Osterberechnung eine konstante Laufzeit hat. Zwar werden durchschnittlich alle 3000 Jahre ein weiterer Schleifendurchlauf bei der Umrechnung nötig, dies kann aber vernachlässigt werden. Die Laufzeit ist also, wenn n die Differenz zwischen Startjahr und Eregbnisjahr darstellt:

\[\mathcal O(n)\]