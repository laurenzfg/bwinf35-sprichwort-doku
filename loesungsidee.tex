\subsection{Definition des Osterdatums}
	\begin{quote}
		Als Osterdatum wurde im Jahre 325 auf dem Konzil von Nicäa der erste Sonntag nach dem ersten Vollmond im Frühling (Datum des Frühlingsvollmondes), der frühestens am 21. März stattfinden kann, festgelegt. Der früheste Ostersonntag fällt folglich auf den 22. März, der späteste auf den 25. April.

		\hfill{}--Wikipedia, https://de.wikipedia.org/wiki/Osterdatum
	\end{quote}

	All diese Bedingungen wurden von Carl Friedrich Gauß in der Gaußschen Osterformel\footnote{\url{https://de.wikipedia.org/wiki/Gau\%C3\%9Fsche_Osterformel}} zusammengefasst. In meinem Programm nutze ich die im Wikipedia-Artikel vorgestellte Fassung von Heiner Lichtenberg.
\subsection{Unterschied zwischen gregorianischem und julianischem Kalender}
	Das Weihnachtsfest wird in beiden Religionen am 25. Dezember gefeiert. Die russisch-orthodoxe Kirche hat allerdings die 1852 durchgeführte Kalenderreform durch Papst Gregor nicht angewandt. Der deshalb dort noch genutzte Vorgängerkalender, der julianische Kalender, stimmt allerdings mit dem heutigen Kalender bis auf die Schaltjahresregelung komplett überein.

	Im julianischen sowie dem gregorianischen Kalender ist jedes durch 4 restlos teilbare Jahr ein Schaltjahr. Allerdings sind im gregorianischem Kalender Jahre, die durch 100 restlos teilbar sind, ausgenommen und damit regelmäßige Jahre. Von dieser Ausnahmeregelung sind wiederum Jahre, die restlos durch 400 teilbar sind ausgenommen und damit doch wieder Schaltjahre.

	Ferner wurden im Reformjahr 1852 10 Kalendertage übersprungen. Deshalb driften die beiden Kalendersysteme alle hundert Jahre um durchschnittlich 0.75 Tage auseinander.  Aus beiden Bedingungen lässt sich folgende Formel\footnote{\url{https://de.wikipedia.org/wiki/Umrechnung_zwischen_julianischem_und_gregorianischem_Kalender}} für den Abstand in Tagen zwischen gregorianischem und julianischem Kalender ableiten, wobei x die Jahreszahl ist. Außerdem sind  wiederum alle Divisionen ohne Rest auszuführen! Für die Monate Januar und Februar gelten Ausnahmen, die für uns aber irrelevant sind (25. Dez. und Ostern zwischen März und April). \textbf{ECHT?}

	\[f(x)=x\div{}100-x\div{}400-2\]

\clearpage
\subsection {Lösungsidee}
	Mit diesen Sachinformationen lässt sich das Problem in recht wenig Zeit mittels Brute Force lösen: Von 2016 an berechnet man alle katholischen Osterfeste und orthodoxen Weihnachtsfeste, bis die Ergebnisse sich gleichen. Eine weitere Optimierung ist nicht möglich, da das Osterdatum unregelmäßig schwankt. Höchstens ist das Caching des orthodoxen Weihnachtsdatums für 100 Jahre möglich. Da also das Überprüfen einer richtigen Lösung sehr schnell geht (es muss nur ein Osterdatum berechnet und umgerechnet werden), das berechnen einer Lösung aber nicht Zeiteffizient möglich ist, ist das Problem \textbf{NP-Vollständig}.

	Das andere Datum lässt sich dementsprechend finden, indem man alle orthodoxen Osterfeste berechnet, bis das Ergebnis dem 25. Dezember des gregorianischen Kalenders entspricht.

	JAHRESOFFSETS!