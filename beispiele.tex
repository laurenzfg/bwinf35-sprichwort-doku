Die Programmausgabe lautet:

\shellcmd{java -jar Sprichwort.jar}
\shellout{Kath./Ev. Ostern und Orth. Weihnachten finden am folgendem Datum gleichzeitig statt: 23.3.11919 \\
Kath./Ev. Weihnachten und Orth. Ostern finden am folgendem Datum gleichzeitig statt: 25.12.32839}

Die Ausgaben habe ich dann noch manuell überprüft:

	\subsection{Kath./Ev. Ostern und Orth. Weihnachten}
		Die Lösung 23 März 11919\textsuperscript{gre}/25. Dez. 11918\textsuperscript{jul} ist eine korrekte Lösung, weil:
		\subsubsection{Kath./Ev. Ostern 11919}
			Nach der Gaußschen Osterformel von 1816.

			\begin{math}
			j = 11919													\\
			a = j \bmod 19 = 6											\\
			b = j \bmod 4 = 3											\\
			c = j \bmod 7 = 3											\\
			k = j \div 100 = 119										\\
			p = (8 \times k + 13) \div 25 = 38							\\
			q = k \div 4 = 29											\\
			M = (15 + k - p -q) \bmod 30 = 7							\\
			d = (19 \times a + M) \bmod 30 = 1							\\
			N = (4 + k - q) \bmod 7 = 3									\\
			e = (2 \times b + 4 \times c + 6 \times d + N) \bmod 7 = 0	\\
			O = 22 + d + e = 23 \text{.März}						
			\end{math}

			Ostern fällt 11919 auf den 23 März\textsuperscript{gre}!
		\clearpage
		\subsubsection{Orthodoxes Weihnachtsdatum 11918}
			Orthodoxes Weihnachtsdatum: 25. Dez. 11918\textsuperscript{jul}. Umrechnung in den gregorianischen Kalender:

			\begin{math}
			JH = 119						\\
			a = JH \div 4 = 29				\\
			b = JH \bmod 4 = 3				\\
			TD = 3 \times a + b - 2 = 88 	\\
			\end{math}

			Vom 25. Dez. 11918 bis zum 1. Jan 11919 sind es 6 Tage, es verbleiben 82 Tage.

			\begin{tabular}{ll}
			Jan 11919: -31                    & R: 51 \\
			Feb 11919: -28 (kein Schaltjahr!) & R: 23
			\end{tabular}

			23 Tage Rest, also liegt der 25. Dez. 11918\textsuperscript{jul} auf dem 23 Mär. 11919\textsuperscript{gre}!
	\clearpage
	\subsection{Kath./Ev. Weihnachten und Orthodoxe Ostern}
		Die Lösung 25. Dez. 32839\textsuperscript{gre}/25. Apr. 32839\textsuperscript{jul} ist eine korrekte Lösung, weil:
		\subsubsection{Kath./Ev. Weihnachtsfest 32839}
			32839 findet Weihnachten natürlich am 25. Dez. 32839\textsuperscript{gre} statt.

		\subsubsection{Orth. Ostern 32839}
			Nach der Gaußschen Osterformel von 1816

			\begin{math}
			j = 32839														\\	
			a = j \bmod 19 = 7												\\	
			b = j \bmod 4 = 3												\\
			c = j \bmod 7 = 3												\\
			k = j \div 100 = 328											\\
			M = (15 + k - p - q) \bmod 30 = 15								\\
			d = (19 \times a + M) \bmod 30 = 28								\\
			N = (4 + k - q) \bmod 7 = 6										\\
			e = (2 \times b + 4 \times c + 6 \times d + N) \bmod 7 = 6		\\
			O = 22 + d + e = 56\text{.März } \widehat{=} \text{ } 25\text{.Apr}
			\end{math}

			Osterdatum : 25. Apr. 32839\textsuperscript{jul}

			Umrechnung in den Gregorianischen Kalender:

			\begin{math}
			JH = 328						\\
			a = JH \div 4 = 82				\\
			b = JH \bmod 4 = 0				\\
			TD = 3 \times a + b - 2 = 244 	\\
			\end{math}

			Vom 25. Apr. 32839 bis zum 1. Mai sind es 5 Tage, es verbleiben 239 Tage:

			\begin{tabular}{ll}
				Mai 32839: -31 & R: 208 \\
				Jun 32839: -30 & R: 178 \\
				Jul 32839: -31 & R: 147 \\
				Aug 32839: -31 & R: 116 \\
				Sep 32839: -30 & R: 86  \\
				Okt 32839: -31 & R: 55  \\
				Nov 32839: -30 & R: 25 
			\end{tabular}

			25 Tage Rest, also liegt der 25. Apr. 32839\textsuperscript{jul} auf dem 25. Dez. 32839\textsuperscript{gre}